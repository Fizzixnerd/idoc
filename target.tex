\documentclass[mpinclude=true]{scrartcl}

\usepackage{amsmath,amsthm,amssymb}

\usepackage[utf8]{inputenc}
\usepackage{scrlayer-notecolumn}
\usepackage{setspace}
\usepackage{hyperref}
\usepackage{braket}

\newtheorem{Theorem}{Theorem}

\setlength{\marginparwidth}{2.67\marginparwidth}
\KOMAoption{headings}{normal}
\KOMAoption{captions}{outerbeside}
\setkomafont{labelinglabel}{\sffamily\bfseries}
\newcommand{\Margin}[1]{\makenote{#1}}

\title{Doc Title}

\subtitle{An \href{https://www.independentlearning.science}{ILS} Document}

\begin{document}
\maketitle
\tableofcontents
\pagestyle{headings}

\section{Section 1 Title}


\subsection{Paragraphs}


This is a paragraph.  You just type like normal.  They are separated
by a blank line.  For example:

This is now a new paragraph.  I can italicize text with \emph{underscores}
and bold it with \textbf{asterisks}.  I can make it \texttt{monospace} using
backticks.

I will end this paragraph with another blank line and then start
talking about lists.

% This is a comment line.  It will not be rendered.

\subsection{Lists}


\begin{itemize}
\item This is an unordered list
\item It will look like bullet points in the final render
\item Third main item
\end{itemize}

\begin{enumerate}
\item This is a numbered list
\item This is the second guy
\item And so on...
\end{enumerate}

\begin{labeling}
{Second Item}\item[First Item] This is a list where the items have labels
\item[Second Item] Another item
\item[And So On] ...
\end{labeling}

\subsection{Math}


Inline math is done just using normal latex by doing $f(x) = \exp
(-x^2)$.  Display mode is done by using a \emph{math block}, like so:

\begin{displaymath}
\label{eq:f-defn} f(x) = \int_x^\infty g(t) dt
\end{displaymath}

Note that that $d$ will not be upright as it should be.  We'll fix
that later.  Also note that we added an \emph{ID} to the equation.  IDs can
be added to many things.  They always appear immediately following the
thing they identify.

\subsection{Blocks}


We can do other types of blocks.  They all look the same, basically.

\begin{quote}
The internet is the most important invention since gravity.
\end{quote}

Somehow later I'll add attribution to the author.  That one is from
Einstein though.

\Margin{This content will be rendered in the margin of the document.  An
equation:$F = \frac{dp}{dt}$}

This will show a little warning symbol next to it.

This is basically a sub-document that is allowed to have extra prereqs

\subsection{Links}


Links look like \href{httpswww.google.com}{this}.  Note that this is
an "outlink" -- a link to an external site -- and so wouldn't actually
be allowed in the main body of the document like this.  We can also
link to \href{Blocks}{headings} in the current document, or
\href{eqn:f-defn}{anything else} which has an ID.  And we can even link
to headings in \href{https://learn.independentlearning.science/Physics/Classical/Mechanics/Newtonian#Second Law}{other documents},
as long as they are a prerequisite of this one.  The same applies to
ID'd items in those docs, including this paragraph. 

\section{Conclusion}


That's about it for the basic syntax.  We can fix the rest later.

\end{document}