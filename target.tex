\documentclass[mpinclude=true]{scrartcl}

\usepackage{amsmath,amsthm,amssymb}

\usepackage[utf8]{inputenc}
\usepackage{scrlayer-notecolumn}
\usepackage{setspace}
\usepackage{hyperref}
\usepackage{braket}

\newtheorem{Theorem}{Theorem}

\setlength{\marginparwidth}{2.67\marginparwidth}
\KOMAoption{headings}{normal}
\KOMAoption{captions}{outerbeside}
\setkomafont{labelinglabel}{\sffamily\bfseries}
\newcommand{\Margin}[1]{\makenote{#1}}

\title{idoc User Guide}

\subtitle{An \href{https://www.independentlearning.science}{ILS} Document}

\begin{document}
\maketitle
\tableofcontents
\pagestyle{headings}

\section{Introduction}


idoc is a language and a program.  The language is what was used to
write this document.  The program is what rendered it (in both LaTeX
and HTML).  It has a number of features that are not found in other
humane markup languages.  The most critical of these is support for
allowing prerequisite resolution, though this is technically
implemented by a static checker after parsing.  idoc was originally
based on asciidoc, but has since diverged significantly.  You will
probably want to examine the source for this document along with the
rendered content.  It can be found
\href{httpwww.independentlearning.science/source.idoc}{here}.

\section{Basics}


\subsection{Paragraphs}


This is a paragraph.  You just type like normal.  They are separated
by a blank line.  For example:

This is now a new paragraph.  I can italicize text with \emph{underscores}
(\texttt{\_}) and bold it with \textbf{asterisks} (\texttt{*}).  I can make it \texttt{monospace}
using backticks (\texttt{`}).  I will end this paragraph with another blank
line.

\subsection{Comments}


Any line that begins with \texttt{//} denotes a comment.  They will be
rendered as comments upon translation, but will not be displayed in
the final document.

% This is a comment line.  It will not be rendered.  Block

% comments are not supported because they are a pain.

\subsection{Lists}


Lists are sometimes useful in documents.  Currently \texttt{idoc} does not
recognize complex content inside lists, though this is subject to
change.  This means you cannot nest lists.  You may only write a
single paragraph.  We support unordered lists, ordered lists and
labelled (description) lists.

\begin{itemize}
\item This is an unordered list.  It can contain paragraph contents.
\item It will look like bullet points in the final render.
\item Third main item.
\end{itemize}

\begin{enumerate}
\item This is a numbered list.
\item This is the second guy.
\item And so on...
\end{enumerate}

\begin{labeling}
{Second Item}\item[First Item] This is a list where the items have labels.
\item[Second Item] Another item.
\item[And So On] ...
\end{labeling}

\subsection{Math}


Inline math is done just using normal latex by doing enclosing text in
dollar signs. $f(x) = \exp (-x^2)$.  Display mode is done by using a
\emph{math block}, like so:

\minisec{Look Ma', I Have Equations!}
\begin{displaymath}
\label{\#eqn:f-defn} f(x) = \int_x^\infty g(t) dt
\end{displaymath}

Note that that $d$ will not be upright as it should be.  We'll fix
that later.  Also note that we added an \emph{ID} to the equation.  IDs can
be added to many things.  They always appear immediately following the
thing they identify.

\subsection{Blocks}


We can do other types of blocks beside math blocks.  They all look the
same, basically.  A block can also have a title, which comes \emph{after}
the declaration of block type.

\minisec{A Blockquote from the "Great One".}
\begin{quote}
\label{\#q:einstein} The internet is the most important invention since gravity.
\end{quote}

Notice above we added an author attribution above as an \emph{attribute} of
the quote block contents.  Attribute lists always come \emph{just} before
the thing they modify.  In this case, we are modifying the "body" of
the \texttt{@ quote} element, so it comes just before the body.

\minisec{Please be Aware:}
\Margin{This content will be rendered in the margin of the document.  You are
allowed any amount of complex content here.  An equation: $F =
\frac{dp}{dt}$.}

This will show a little warning symbol next to it.

\minisec{An aside}
\minisec{Aside Prerex}


An aside is useful when you have content connected to the main text
but which isn't super necessary, or requires more advanced techniques
than the rest of the text.This is basically a sub-document that is allowed to have extra
prereqs.(Sub)sections are \emph{not} allowed here, since these are meant to be short.

\subsection{Links}


Links look like \href{httpswww.google.com}{this}.  Note that this is
an "outlink" -- a link to an external site -- and so wouldn't actually
be allowed in the main body of the document like this.  We can also
link to \href{Introduction}{headings} in the current document, or
\href{eqn:f-defn}{anything else} which has an ID.  And we can even link
to headings in \href{https://learn.independentlearning.science/Physics/Classical/Mechanics/Newtonian#Second Law}{other documents},
as long as they are a prerequisite of this one.  The same applies to
ID'd items in those docs, including this paragraph. 

\section{Conclusion}


That's about it for the basic syntax.  We can fix the rest later.

\end{document}