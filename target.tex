\documentclass[12pt]{article}

\usepackage{amsmath,amsthm,amssymb}
\usepackage{geometry}
\usepackage{hyperref}
\usepackage{braket}

\newtheorem{Theorem}{Theorem}
\newtheorem{Example}{Example}

 \title{Doc Title}


 \begin{document}
\maketitle
\tableofcontents

 \section{Section 1 Title}


 \subsection{Paragraphs}


This is a paragraph.  You just type like normal.  They are separated by a blank line.  For example:

This is now a new paragraph.  I can italicize text with \emph{underscores}

and bold it with \textbf{asterisks}.  I can make it `monospace` using backticks.

I will end this paragraph with another blank line and then start talking about lists.

// This is a comment line.  It will not be rendered.

 \subsection{Lists}


 \begin{itemize}
 \item This is an unordered list - It will look like bullet points in the final render - Third main item \end{itemize}


 \begin{enumerate}
 \item This is a numbered list . This is the second guy . And so on... \end{enumerate}


First Item:: This is a list where the items have labels Second Item:: Another item And So On:: ...

 \subsection{Math}


Inline math is done just using normal latex by doing $f(x) = \exp
(-x^2)$.  Display mode is done by using a \emph{math block}, like so:


 \label{eqn:f-defn}\[f(x) = \int_x^\infty g(t) dt
\]

Note that that $d$ will not be upright as it should be.  We'll fix that later.  Also note that we added an \emph{ID} to the equation.  IDs can be added to many things.  They always appear immediately following the thing they identify.

 \subsection{Blocks}


We can do other types of blocks.  They all look the same, basically.


\[The internet is the most important invention since gravity.
\]

Somehow later I'll add attribution to the author.  That one is from Einstein though.


\[This will show a little warning symbol next to it.
\]


\[This is basically a sub-document that is allowed to have extra prereqs
\]

 \subsection{Links}


Links look like \href{https://learn.independentlearning.science/https://www.google.com}{this}.  Note that this is an "outlink" -- a link to an external site -- and so wouldn't actually be allowed in the main body of the document like this.  We can also link to \href{Blocks}{headings} in the current document, or \href{eqn:f-defn}{anything else} which has an ID.  And we can even link to headings in \href{/Physics/Classical/Mechanics/Newtonian#Second Law}{other documents}, as long as they are a prerequisite of this one.  The same applies to ID'd items in those docs, including this paragraph.

 \section{Conclusion}


That's about it for the basic syntax.  We can fix the rest later. \end{document}
